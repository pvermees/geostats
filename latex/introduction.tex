\chapter{Introduction}

According to the Oxford dictionary of English, the definition of
`statistics' is:

\begin{quote}
  \textit{The practice or science of collecting and analysing
    numerical data in large quantities, especially for the purpose of
    \textbf{inferring} proportions in a whole from those in a
    representative \textbf{sample}.}
\end{quote}

The words `inferring' and `sample' are written in bold face because
they are really central to the practice and purpose of Science in
general, and Geology as a whole. For example:

\begin{enumerate}
\item The true proportion of quartz grains in a sand deposit is
  \emph{unknown} but can be \emph{estimated} by counting a number of
  grains from a representative sample.
\item The true crystallisation age of a rock is unknown but can be
  estimated by measuring the
  \textsuperscript{206}Pb/\textsuperscript{238}U-ratio in a
  representative number of U-bearing mineral grains from that rock.
\item The true \textsuperscript{206}Pb/\textsuperscript{238}U-ratio of
  a U-bearing mineral is unknown but can be estimated by repeatedly
  measuring the ratio of \textsuperscript{206}Pb- and
  \textsuperscript{238}U- ions extracted from that mineral in a mass
  spectrometer.
\item The spatial distribution of arsenic in groundwater is unknown
  but can be estimated by measuring the arsenic content of a finite
  number of water wells.
\end{enumerate}

Thus, pretty much everything that we do as Earth Scientists involves
statistics in one way or another. This module will introduce you to
some basic principles of statistics, before moving on to `geological'
data. The main purpose of the module is to instill a critical attitude
in the student, and an awareness of the many pitfalls of blindly
applying statistical `black boxes' to geological problems. The module
takes a hands-on approach, using a popular statistical programming
language called \texttt{R} (Chapter~\ref{ch:R}).
