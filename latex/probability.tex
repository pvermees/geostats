\chapter{Probability}
\label{ch:probability}

Probability is a numerical description of how likely it is for an
event to occur or how likely it is for a proposition to be true.  In
the context of a sampling experiment, the probability of an outcome
$A$ can be defined as:

\begin{equation}
P(A) = \frac{\mbox{the number of ways }A\mbox{ can occur}}
{\mbox{the total number of outcomes}}
\end{equation}

For example, the probability of tossing an unbiased coin and observing
a head ($H$) is

\[
\frac{
  \{H\}
}{
  \{H\}\{T\}
} = \frac{1}{2} = 0.5
\]

The probability of tossing the same coin three times and obtaining two
heads and one tail ($T$) is:

\begin{equation}
  P\!\left(2\times{H} \cap 1\times{T}\right) =
  \frac{
    \{THH\}\{HTH\}\{HHT\}
  }{
    \{HHH\}\{THH\}\{HTH\}\{HHT\}\{TTH\}\{THT\}\{HTT\}\{TTT\}
  } = \frac{3}{8}
  \label{eq:2H1T}
\end{equation}

\noindent where $\cap$ stands for `and' in the sense of ``the
intersection of two sets of outcomes''. Similarly, the probability of
throwing two dice and obtaining a two (\raisebox{-2pt}{\Cube{2}}) and
a six (\raisebox{-2pt}{\Cube{6}}) is:
  
\begin{equation}
  P\!\left(1\times{\mbox{\raisebox{-2pt}{\Cube{2}}}} \cap
  1\times\mbox{\raisebox{-2pt}{\Cube{6}}}\right) =
  \frac{
    \mbox{
      \big\{\raisebox{-2pt}{\Cube{2} \Cube{6}}\big\}\big\{\raisebox{-2pt}{\Cube{6} \Cube{2}}\big\}
    }
  }{
    \begin{array}{c}
      \mbox{
        \big\{\raisebox{-2pt}{\Cube{1} \Cube{1}}\big\}\big\{\raisebox{-2pt}{\Cube{2} \Cube{1}}\big\}\big\{\raisebox{-2pt}{\Cube{3} \Cube{1}}\big\}\big\{\raisebox{-2pt}{\Cube{4} \Cube{1}}\big\}\big\{\raisebox{-2pt}{\Cube{5} \Cube{1}}\big\}\big\{\raisebox{-2pt}{\Cube{6} \Cube{1}}\big\}
      }\\
      \mbox{
        \big\{\raisebox{-2pt}{\Cube{1} \Cube{2}}\big\}\big\{\raisebox{-2pt}{\Cube{2} \Cube{2}}\big\}\big\{\raisebox{-2pt}{\Cube{3} \Cube{2}}\big\}\big\{\raisebox{-2pt}{\Cube{4} \Cube{2}}\big\}\big\{\raisebox{-2pt}{\Cube{5} \Cube{2}}\big\}\big\{\raisebox{-2pt}{\Cube{6} \Cube{2}}\big\}
      }\\
      \mbox{
        \big\{\raisebox{-2pt}{\Cube{1} \Cube{3}}\big\}\big\{\raisebox{-2pt}{\Cube{2} \Cube{3}}\big\}\big\{\raisebox{-2pt}{\Cube{3} \Cube{3}}\big\}\big\{\raisebox{-2pt}{\Cube{4} \Cube{3}}\big\}\big\{\raisebox{-2pt}{\Cube{5} \Cube{3}}\big\}\big\{\raisebox{-2pt}{\Cube{6} \Cube{3}}\big\}
      }\\
      \mbox{
        \big\{\raisebox{-2pt}{\Cube{1} \Cube{4}}\big\}\big\{\raisebox{-2pt}{\Cube{2} \Cube{4}}\big\}\big\{\raisebox{-2pt}{\Cube{3} \Cube{4}}\big\}\big\{\raisebox{-2pt}{\Cube{4} \Cube{4}}\big\}\big\{\raisebox{-2pt}{\Cube{5} \Cube{4}}\big\}\big\{\raisebox{-2pt}{\Cube{6} \Cube{4}}\big\}
      }\\
      \mbox{
        \big\{\raisebox{-2pt}{\Cube{1} \Cube{5}}\big\}\big\{\raisebox{-2pt}{\Cube{2} \Cube{5}}\big\}\big\{\raisebox{-2pt}{\Cube{3} \Cube{5}}\big\}\big\{\raisebox{-2pt}{\Cube{4} \Cube{5}}\big\}\big\{\raisebox{-2pt}{\Cube{5} \Cube{5}}\big\}\big\{\raisebox{-2pt}{\Cube{6} \Cube{5}}\big\}
      }\\
      \mbox{
        \big\{\raisebox{-2pt}{\Cube{1} \Cube{6}}\big\}\big\{\raisebox{-2pt}{\Cube{2} \Cube{6}}\big\}\big\{\raisebox{-2pt}{\Cube{3} \Cube{6}}\big\}\big\{\raisebox{-2pt}{\Cube{4} \Cube{6}}\big\}\big\{\raisebox{-2pt}{\Cube{5} \Cube{6}}\big\}\big\{\raisebox{-2pt}{\Cube{6} \Cube{6}}\big\}
      }
    \end{array}
  } = \frac{2}{36} = \frac{1}{18}
  \label{eq:1116}
\end{equation}

The \textbf{multiplicative rule of
  probability}\label{page:multiplication} dictates that the
probability of two combined (independent) experiments is given by the
product of their respective probabilities. Thus, if one were to carry
out a coin tossing \emph{and} a dice throwing experiment, then the
probability of obtaining two heads and one tail for the first
experiment \emph{and} throwing a two and a six in the second
experiment is:

\[
P\left(\left\{2\times{H} \cap 1\times{T}\right\} \cap
\left\{1\times\mbox{\raisebox{-2pt}{\Cube{2}}} \cap
1\times\mbox{\raisebox{-2pt}{\Cube{6}}}\right\}\right) =
\frac{3}{8}\frac{1}{16} = \frac{3}{128} = 0.023
\]

The \textbf{additive rule of probability}\label{page:addition}
dictates that the probability of observing either of two multually
exclusive outcomes is given by the sum of their respective
probabilities. Thus, if one were to carry out a coin tossing and a
dice throwing experiment, then the probability of obtaining two heads
and one tail for the first experiment \emph{or} throwing a two and a
six in the second experiment is:

\[
P\left(\left\{2\times{H} \cap 1\times{T}\right\} \cup
\left\{1\times\mbox{\raisebox{-2pt}{\Cube{2}}} \cap
1\times\mbox{\raisebox{-2pt}{\Cube{6}}}\right\}\right) =
\frac{3}{8} + \frac{1}{16} = \frac{7}{16} = 0.4375
\]

\section{Permutations}
\label{sec:permutations}

A permutation is an ordered arrangement of objects. These objects can
be selected in one of two ways:

\begin{enumerate}
\item{\bf sampling with replacement.} Consider an urn with $n$ balls
  that are numbered 1 through $n$. Draw a ball from the urn and write
  down its number. There are $n$ possible outcomes for this
  experiment.  Then place the ball back in the urn, thoroughly mix the
  balls and draw a second one. Write down its number. There are $n$
  possible outcomes for the second experiment. Using the
  multiplicative rule of probability, there are $(n\times{n})$
  possible outcomes for the two numbers. Repeat until you have drawn
  $k$ balls (where $k \leq n$). Then the number of possible
  combinations of numbers is
  \begin{equation}
  \overset{k \mbox{~times}}{\overbrace{n\times{n}\times{\ldots}\times{n}}} = n^k
  \end{equation}
  
\item{\bf sampling without replacement.} Consider the same urn as
  before and draw a first number. Like before, there are $n$ possible
  outcomes.  However this time we do not put the ball back into the
  urn. With the first ball removed, draw a second number. This time
  there are only $(n - 1)$ possible outcomes. Thus, the total number
  of combinations for the first two numbers is $n \times (n-1)$. Continuing
  the experiment until you have drawn $k$ balls yields 
  \begin{equation}
  n\times(n-1)\times(n-2)\times{\ldots}\times(n-k+1) = \frac{n!}{(n-k)!}
  \end{equation}
  possible combinations, where `!' is the factorial operator.

\end{enumerate}

Let us apply these two formulas to a classical statistical problem:
``\textit{what is the probability that two students in a classroom of
  $r$ celebrate their birthdays on the same day?}''. The solution is
as follows.

\begin{enumerate}
\item There are ($n=$) 365 possible days on which the first person might
  celebrate their birthday.
\item There are 365 possible days on which the second person might
  celebrate their birthday, but only 364 of these do not overlap with
  the first person's birthday.
\item There are another 365 possible days on which the third person
  might celebrate their birthday, but only 363 of these do not overlap
  with the birthdays of the first two people.
\item For $k$ people, there are $365^k$ possible combinations of
  birthdays (sampling with replacement), but only
  $365\times{364}\times{\ldots}(k+1) = 365!/(365-k)!$ of these combinations
  do not overlap (sampling without replacement).
\item Therefore, the probability that two people's birthdays do
  not overlap is given by
  \[
  P(\mbox{no overlapping birthdays}) = \frac{365!}{(365-k)!365^k}
  \]
\item And the probability that at least two people's birthdays
  overlap is
  \[
  P(>1~\mbox{overlapping birthdays}) = 1 - \frac{365!}{(365-k)!365^k}
  \]  
\end{enumerate}

If $k = 23$, then $P(>1~\mbox{overlapping birthdays}) = 0.507$. In
other words, there is a greater than 50\% chance that at least two
students will share the same birthday in a classroom of 23.

\section{Combinations}
\label{sec:combinations}

Section~\ref{sec:permutations} showed that there are $n^k$ unique
possible ways to select $k$ objects from a collection of $n$ with
replacement. For example, there are $2^3=8$ ways for three coins to
land:

\[\{HHH\}\{THH\}\{HTH\}\{HHT\}\{TTH\}\{THT\}\{HTT\}\{TTT\}\]

\noindent and there are $6^2=36$ ways for two dice to land:

\[
\begin{array}{c}
\mbox{ \big\{\raisebox{-2pt}{\Cube{1}
    \Cube{1}}\big\}\big\{\raisebox{-2pt}{\Cube{2}
    \Cube{1}}\big\}\big\{\raisebox{-2pt}{\Cube{3}
    \Cube{1}}\big\}\big\{\raisebox{-2pt}{\Cube{4}
    \Cube{1}}\big\}\big\{\raisebox{-2pt}{\Cube{5}
    \Cube{1}}\big\}\big\{\raisebox{-2pt}{\Cube{6} \Cube{1}}\big\}
}\\ \mbox{ \big\{\raisebox{-2pt}{\Cube{1}
    \Cube{2}}\big\}\big\{\raisebox{-2pt}{\Cube{2}
    \Cube{2}}\big\}\big\{\raisebox{-2pt}{\Cube{3}
    \Cube{2}}\big\}\big\{\raisebox{-2pt}{\Cube{4}
    \Cube{2}}\big\}\big\{\raisebox{-2pt}{\Cube{5}
    \Cube{2}}\big\}\big\{\raisebox{-2pt}{\Cube{6} \Cube{2}}\big\}
}\\ \mbox{ \big\{\raisebox{-2pt}{\Cube{1}
    \Cube{3}}\big\}\big\{\raisebox{-2pt}{\Cube{2}
    \Cube{3}}\big\}\big\{\raisebox{-2pt}{\Cube{3}
    \Cube{3}}\big\}\big\{\raisebox{-2pt}{\Cube{4}
    \Cube{3}}\big\}\big\{\raisebox{-2pt}{\Cube{5}
    \Cube{3}}\big\}\big\{\raisebox{-2pt}{\Cube{6} \Cube{3}}\big\}
}\\ \mbox{ \big\{\raisebox{-2pt}{\Cube{1}
    \Cube{4}}\big\}\big\{\raisebox{-2pt}{\Cube{2}
    \Cube{4}}\big\}\big\{\raisebox{-2pt}{\Cube{3}
    \Cube{4}}\big\}\big\{\raisebox{-2pt}{\Cube{4}
    \Cube{4}}\big\}\big\{\raisebox{-2pt}{\Cube{5}
    \Cube{4}}\big\}\big\{\raisebox{-2pt}{\Cube{6} \Cube{4}}\big\}
}\\ \mbox{ \big\{\raisebox{-2pt}{\Cube{1}
    \Cube{5}}\big\}\big\{\raisebox{-2pt}{\Cube{2}
    \Cube{5}}\big\}\big\{\raisebox{-2pt}{\Cube{3}
    \Cube{5}}\big\}\big\{\raisebox{-2pt}{\Cube{4}
    \Cube{5}}\big\}\big\{\raisebox{-2pt}{\Cube{5}
    \Cube{5}}\big\}\big\{\raisebox{-2pt}{\Cube{6} \Cube{5}}\big\}
}\\ \mbox{ \big\{\raisebox{-2pt}{\Cube{1}
    \Cube{6}}\big\}\big\{\raisebox{-2pt}{\Cube{2}
    \Cube{6}}\big\}\big\{\raisebox{-2pt}{\Cube{3}
    \Cube{6}}\big\}\big\{\raisebox{-2pt}{\Cube{4}
    \Cube{6}}\big\}\big\{\raisebox{-2pt}{\Cube{5}
    \Cube{6}}\big\}\big\{\raisebox{-2pt}{\Cube{6} \Cube{6}}\big\}
}
\end{array}\]
    
Note that these are the denumenators of the two examples that were
given in the introductory paragraph of this chapter. Also note that
the collection of outcomes contains quite a few duplicate values.  For
the coin example, these fall into two groups of three duplicates:
\[\{\{HTT\}\{THT\}\{HHT\}\} \mbox{~and~} \{\{THH\}\{HTH\}\{TTH\}\}\]

For the dice, there are fifteen groups of duplicate pairs:
\[
\begin{array}{l}
  \big\{
  \big\{\raisebox{-2pt}{\Cube{1} \Cube{2}}\big\}
  \big\{\raisebox{-2pt}{\Cube{2} \Cube{1}}\big\}
  \big\}
  \big\{
  \big\{\raisebox{-2pt}{\Cube{1} \Cube{3}}\big\}
  \big\{\raisebox{-2pt}{\Cube{3} \Cube{1}}\big\}
  \big\}
  \big\{
  \big\{\raisebox{-2pt}{\Cube{1} \Cube{4}}\big\}
  \big\{\raisebox{-2pt}{\Cube{4} \Cube{1}}\big\}
  \big\}
  \big\{
  \big\{\raisebox{-2pt}{\Cube{1} \Cube{5}}\big\}
  \big\{\raisebox{-2pt}{\Cube{5} \Cube{1}}\big\}
  \big\}
  \big\{
  \big\{\raisebox{-2pt}{\Cube{1} \Cube{6}}\big\}
  \big\{\raisebox{-2pt}{\Cube{6} \Cube{1}}\big\}
  \big\}\\
  \big\{
  \big\{\raisebox{-2pt}{\Cube{2} \Cube{3}}\big\}
  \big\{\raisebox{-2pt}{\Cube{3} \Cube{2}}\big\}
  \big\}
  \big\{
  \big\{\raisebox{-2pt}{\Cube{2} \Cube{4}}\big\}
  \big\{\raisebox{-2pt}{\Cube{4} \Cube{2}}\big\}
  \big\}
  \big\{
  \big\{\raisebox{-2pt}{\Cube{2} \Cube{5}}\big\}
  \big\{\raisebox{-2pt}{\Cube{5} \Cube{2}}\big\}
  \big\}
  \big\{
  \big\{\raisebox{-2pt}{\Cube{2} \Cube{6}}\big\}
  \big\{\raisebox{-2pt}{\Cube{6} \Cube{2}}\big\}
  \big\}
  \big\{
  \big\{\raisebox{-2pt}{\Cube{3} \Cube{4}}\big\}
  \big\{\raisebox{-2pt}{\Cube{4} \Cube{3}}\big\}
  \big\}\\
  \big\{
  \big\{\raisebox{-2pt}{\Cube{3} \Cube{5}}\big\}
  \big\{\raisebox{-2pt}{\Cube{5} \Cube{3}}\big\}
  \big\}
  \big\{
  \big\{\raisebox{-2pt}{\Cube{3} \Cube{6}}\big\}
  \big\{\raisebox{-2pt}{\Cube{6} \Cube{3}}\big\}
  \big\}
  \big\{
  \big\{\raisebox{-2pt}{\Cube{4} \Cube{5}}\big\}
  \big\{\raisebox{-2pt}{\Cube{5} \Cube{4}}\big\}
  \big\}
  \big\{
  \big\{\raisebox{-2pt}{\Cube{4} \Cube{6}}\big\}
  \big\{\raisebox{-2pt}{\Cube{6} \Cube{4}}\big\}
  \big\}
  \big\{
  \big\{\raisebox{-2pt}{\Cube{5} \Cube{6}}\big\}
  \big\{\raisebox{-2pt}{\Cube{6} \Cube{5}}\big\}
  \big\}
\end{array}
\]

Suppose that we don't care in which order the objects (coins, dice)
appear. How many different \textbf{unordered} samples are possible?

\begin{center}
  (\# ordered samples) =
  (\# unordered samples) $\times$ (\# ways to order the samples)
\end{center}

There are $n!/(n-k)!$ ways to select $k$ objects from a collection of
$n$, and there are $k!$ ways to order these $k$ objects. Therefore

\[
\mbox{
  (\# unordered samples)} =
\frac{\mbox{(\# ordered samples)}}{\mbox{(\# ways to order the samples)}
} =
\frac{n!}{(n-k)!k!}
\]

The formula on the right hand side of this equation gives the number
of combinations of $k$ elements among a collection of $n$.  This
formula is also known as the \textbf{binomial coefficient} and is
often written as $\binom{n}{k}$ (pronounce ``n choose k''):

\begin{equation}
  \binom{n}{k} = \frac{n!}{(n-k)!k!}
  \label{eq:nchoosek}
\end{equation}

Revisiting the two examples at the start of this chapter, the number
of ways to arrange two heads among three coins is
\[
\binom{3}{2} = \frac{3!}{1!2!} = \frac{6}{2} = 3
\]

\noindent which is the numerator of Equation~\ref{eq:2H1T}; and the
number of combinations of one $\raisebox{-2pt}{\Cube{1}}$ and one
$\raisebox{-2pt}{\Cube{6}}$ is
\[
\binom{1}{2} = \frac{2!}{1!1!} = 2
\]

\noindent which is the numerator of Equation~\ref{eq:1116}.

\section{Conditional probability}
\label{sec:conditionalprobability}

So far we have assumed that all experiments (coin tosses, throws of a
dice) were done \emph{independently}, so that the outcome of one
experiment did not affect that of the other. However this is not
always the case in geology. Sometimes one event depends on another
one. We can capture this phenomenon with the following definition:
\begin{equation}
P(A|B) = \mbox{``The conditional probability of}~A~\mbox{given}~B\mbox{''}
\end{equation}

Let $P(A)$ be the probability that a sedimentary deposit contains
\emph{ammonite} fossils.  And let $P(B)$ be the proportion of our
field area that is covered by sedimentary rocks of \emph{Bajocian} age
(170.3 -- 168.3~Ma). Then $P(A|B)$ is the probability that a given
Bajocian deposit contains ammonite fossils. Conversely, $P(B|A)$ is
the probability that an ammonite fossil came from a Bajocian
deposit.\\

The \textbf{multiplication law} dictates that:

\begin{equation}
  P({A}\cap{B}) = P(A|B) P(B) = P(B|A) P(A) = P({B}\cap{A})
  \label{eq:PA&B}
\end{equation}

Suppose that 70\% of our field area is covered by Bajocian deposits
($P(B)=0.7$), and that 20\% of those Bajocian deposits contain
ammonite fossils ($P(A|B)=0.2$). Then there is a 14\%
($=0.7\times{0.2}$) chance that the field area contains Bajocian
ammonites.\\

The \textbf{law of total probability} prescribes that, given $n$
mutually exclusive scenarios $B_i$ (for $1\leq{i}\leq{n}$):
\begin{equation}
  P(A) = \sum\limits_{i=1}^{n} P(A|B_i) P(B_i)
  \label{eq:totalprob}
\end{equation}

Consider a river whose catchment contains 70\% Bajocian deposits
($P(B_1)=0.8$) and 30\% \emph{Bathonian}\footnote{the Bathonian
  immediately overlies the Bajocian and is dated 168.3 -- 166.1~Ma}
deposits ($P(B_2)=0.2$). Recall that the Bajocian is 20\% likely to
contain ammonite fossils ($P(A|B_1)=0.2$), and suppose that the
Bathonian is 50\% likely to contain such fossils ($P(A|B_2)=0.5$).
How likely is it that the river catchment contains ammonites? Using
Equation~\ref{eq:totalprob}:
\[
P(A) = P(A|B_1) P(B_1) + P(A|B_2) P(B_2) =
0.2 \times 0.7 + 0.5 \times 0.3 = 0.29
\]

Equation~\ref{eq:PA&B} can be rearranged to form \textbf{Bayes' Rule}:
\begin{equation}
  P(B|A) = \frac{P(A|B) P(B)}{P(A)}
  \label{eq:Bayes}
\end{equation}

\noindent or, combining Equation~\ref{eq:Bayes} with
Equation~\ref{eq:totalprob}:
\begin{equation}
  P(B_j|A) = \frac{P(A|B_j) P(B_j)}{\sum\limits_{i=1}^{n}P(A|B_i) P(B_i)}
  \label{eq:totalBayes}
\end{equation}

Suppose that we have found an ammonite fossil in the river bed.  What
is its likely age? Using Equation~\ref{eq:totalBayes}, the probability
that the unknown fossil is Bajocian ($B_1$) is given by:
\[
  P(B_1|A) = \frac{P(A|B_1) P(B_1)}{P(A|B_1) P(B_1) + P(A|B_2) P(B_2)} = 
   \frac{0.2 \times 0.7}{0.2 \times 0.7 + 0.5 \times 0.3} = 0.48
\]

Thus there is a 48\% chance that the fossil is Bajocian, and a 52\%
chance that it is Bathonian.
