\documentclass[letterpaper]{book}
\usepackage[times,inconsolata,hyper]{Rd}
\usepackage{makeidx}
\usepackage[utf8]{inputenc} % @SET ENCODING@
% \usepackage{graphicx} % @USE GRAPHICX@
\makeindex{}
\begin{document}
\chapter*{}
\begin{center}
{\textbf{\huge Package `geostats'}}
\par\bigskip{\large \today}
\end{center}
\begin{description}
\raggedright{}
\inputencoding{utf8}
\item[Title]\AsIs{An Introduction to Statistics for Geoscientists}
\item[Version]\AsIs{1.0}
\item[Date]\AsIs{2020-10-05}
\item[Description]\AsIs{Provides example datasets and code for the introductory statistics module for geoscientists at University College London (UCL). Includes functionality for compositional data, spatial data, fractals and chaos.}
\item[Author]\AsIs{Pieter Vermeesch [aut, cre]}
\item[Maintainer]\AsIs{Pieter Vermeesch }\email{p.vermeesch@ucl.ac.uk}\AsIs{}
\item[Depends]\AsIs{R (>= 3.0.0)}
\item[Imports]\AsIs{MASS, grDevices}
\item[License]\AsIs{GPL-3}
\item[URL]\AsIs{}\url{http://ucl.ac.uk/~ucfbpve/isoplotr/}\AsIs{}
\item[LazyData]\AsIs{true}
\item[RoxygenNote]\AsIs{7.1.1}
\item[Encoding]\AsIs{UTF-8}
\end{description}
\Rdcontents{\R{} topics documented:}
